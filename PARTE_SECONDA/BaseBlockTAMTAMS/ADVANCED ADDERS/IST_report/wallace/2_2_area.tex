\subsection{Occupied Area}
In order to evaluate the occupied area of the Wallace Tree, the number of full adders, half adders and D flip flops have been considered. The adders have been implemented by using exclusively ND2 gate while flip flops also include inverter gates.

In Figures \ref{fig:half_adder}, \ref{fig:full_adder} and \ref{fig:f_f} are shown the designs of an half adder, a full adder and D flip flops implemented as previously described.

\begin{figure}[H]
	\centering
	\textsf{{\fontsize{9pt}{5.5pt}
			\begin{pgfpicture}{0cm}{0cm}{217pt}{113pt}
% Created by FidoCadJ ver. 0.24.6, export filter by Davide Bucci
\pgfsetxvec{\pgfpoint{1pt}{0pt}}
\pgfsetyvec{\pgfpoint{0pt}{1pt}}
\pgfsetroundjoin 
\pgfsetroundcap
\pgftranslateto{\pgfxy(0,113)}
\begin{pgfmagnify}{1}{-1}
% Layer color definitions
\definecolor{layer0}{rgb}{0.0,0.0,0.0}
\definecolor{layer1}{rgb}{0.0,0.0,0.5}
\definecolor{layer2}{rgb}{1.0,0.0,0.0}
\definecolor{layer3}{rgb}{0.0,0.5,0.5}
\definecolor{layer4}{rgb}{1.0,0.78,0.0}
\definecolor{layer5}{rgb}{0.5,1.0,0.0}
\definecolor{layer6}{rgb}{0.0,1.0,1.0}
\definecolor{layer7}{rgb}{0.0,0.5,0.0}
\definecolor{layer8}{rgb}{0.6,0.8,0.2}
\definecolor{layer9}{rgb}{1.0,0.08,0.58}
\definecolor{layer10}{rgb}{0.71,0.61,0.05}
\definecolor{layer11}{rgb}{0.0,0.5,1.0}
\definecolor{layer12}{rgb}{0.88,0.88,0.88}
\definecolor{layer13}{rgb}{0.64,0.64,0.64}
\definecolor{layer14}{rgb}{0.37,0.37,0.37}
\definecolor{layer15}{rgb}{0.0,0.0,0.0}
% End of color definitions
\color{layer0}
\pgfsetlinewidth{0.7pt}
\pgfline{\pgfxy(33.0,31.0)}{\pgfxy(40.0,31.0)}
\pgfline{\pgfxy(40.0,28.0)}{\pgfxy(40.0,48.0)}
\pgfline{\pgfxy(33.0,45.0)}{\pgfxy(40.0,45.0)}
\pgfline{\pgfxy(40.0,48.0)}{\pgfxy(51.0,48.0)}
\pgfline{\pgfxy(40.0,28.0)}{\pgfxy(51.0,28.0)}
\pgfmoveto{\pgfxy(51,28)} 
\pgfcurveto{\pgfxy(64,28)}{\pgfxy(64,48)}{\pgfxy(51,48)}
\pgfstroke
\pgfellipse[stroke]{\pgfxy(62.5,38.0)}{\pgfxy(1.5,0)}{\pgfxy(0,1.0)}
\pgfline{\pgfxy(64.0,38.0)}{\pgfxy(68.0,38.0)}
\pgfline{\pgfxy(82.0,10.0)}{\pgfxy(89.0,10.0)}
\pgfline{\pgfxy(89.0,7.0)}{\pgfxy(89.0,27.0)}
\pgfline{\pgfxy(82.0,24.0)}{\pgfxy(89.0,24.0)}
\pgfline{\pgfxy(89.0,27.0)}{\pgfxy(100.0,27.0)}
\pgfline{\pgfxy(89.0,7.0)}{\pgfxy(100.0,7.0)}
\pgfmoveto{\pgfxy(100,7)} 
\pgfcurveto{\pgfxy(113,7)}{\pgfxy(113,27)}{\pgfxy(100,27)}
\pgfstroke
\pgfellipse[stroke]{\pgfxy(111.5,17.0)}{\pgfxy(1.5,0)}{\pgfxy(0,1.0)}
\pgfline{\pgfxy(113.0,17.0)}{\pgfxy(117.0,17.0)}
\pgfline{\pgfxy(82.0,52.0)}{\pgfxy(89.0,52.0)}
\pgfline{\pgfxy(89.0,49.0)}{\pgfxy(89.0,69.0)}
\pgfline{\pgfxy(82.0,66.0)}{\pgfxy(89.0,66.0)}
\pgfline{\pgfxy(89.0,69.0)}{\pgfxy(100.0,69.0)}
\pgfline{\pgfxy(89.0,49.0)}{\pgfxy(100.0,49.0)}
\pgfmoveto{\pgfxy(100,49)} 
\pgfcurveto{\pgfxy(113,49)}{\pgfxy(113,69)}{\pgfxy(100,69)}
\pgfstroke
\pgfellipse[stroke]{\pgfxy(111.5,59.0)}{\pgfxy(1.5,0)}{\pgfxy(0,1.0)}
\pgfline{\pgfxy(113.0,59.0)}{\pgfxy(117.0,59.0)}
\pgfline{\pgfxy(82.0,94.0)}{\pgfxy(89.0,94.0)}
\pgfline{\pgfxy(89.0,91.0)}{\pgfxy(89.0,111.0)}
\pgfline{\pgfxy(82.0,108.0)}{\pgfxy(89.0,108.0)}
\pgfline{\pgfxy(89.0,111.0)}{\pgfxy(100.0,111.0)}
\pgfline{\pgfxy(89.0,91.0)}{\pgfxy(100.0,91.0)}
\pgfmoveto{\pgfxy(100,91)} 
\pgfcurveto{\pgfxy(113,91)}{\pgfxy(113,111)}{\pgfxy(100,111)}
\pgfstroke
\pgfellipse[stroke]{\pgfxy(111.5,101.0)}{\pgfxy(1.5,0)}{\pgfxy(0,1.0)}
\pgfline{\pgfxy(113.0,101.0)}{\pgfxy(117.0,101.0)}
\pgfline{\pgfxy(131.0,31.0)}{\pgfxy(138.0,31.0)}
\pgfline{\pgfxy(138.0,28.0)}{\pgfxy(138.0,48.0)}
\pgfline{\pgfxy(131.0,45.0)}{\pgfxy(138.0,45.0)}
\pgfline{\pgfxy(138.0,48.0)}{\pgfxy(149.0,48.0)}
\pgfline{\pgfxy(138.0,28.0)}{\pgfxy(149.0,28.0)}
\pgfmoveto{\pgfxy(149,28)} 
\pgfcurveto{\pgfxy(162,28)}{\pgfxy(162,48)}{\pgfxy(149,48)}
\pgfstroke
\pgfellipse[stroke]{\pgfxy(160.5,38.0)}{\pgfxy(1.5,0)}{\pgfxy(0,1.0)}
\pgfline{\pgfxy(162.0,38.0)}{\pgfxy(166.0,38.0)}
\pgfline{\pgfxy(68.0,38.0)}{\pgfxy(75.0,38.0)}
\pgfline{\pgfxy(75.0,38.0)}{\pgfxy(75.0,24.0)}
\pgfline{\pgfxy(75.0,24.0)}{\pgfxy(82.0,24.0)}
\pgfline{\pgfxy(75.0,38.0)}{\pgfxy(75.0,52.0)}
\pgfline{\pgfxy(75.0,52.0)}{\pgfxy(82.0,52.0)}
\pgfsetlinewidth{0.33pt}
\pgfcircle[fill]{\pgfxy(75,38)}{1.4pt}\pgfsetlinewidth{0.7pt}
\pgfline{\pgfxy(68.0,101.0)}{\pgfxy(75.0,101.0)}
\pgfline{\pgfxy(75.0,101.0)}{\pgfxy(75.0,94.0)}
\pgfline{\pgfxy(75.0,94.0)}{\pgfxy(82.0,94.0)}
\pgfline{\pgfxy(75.0,101.0)}{\pgfxy(75.0,108.0)}
\pgfline{\pgfxy(75.0,108.0)}{\pgfxy(82.0,108.0)}
\pgfsetlinewidth{0.33pt}
\pgfcircle[fill]{\pgfxy(75,101)}{1.4pt}\pgfsetlinewidth{0.7pt}
\pgfline{\pgfxy(12.0,10.0)}{\pgfxy(82.0,10.0)}
\pgfline{\pgfxy(12.0,66.0)}{\pgfxy(82.0,66.0)}
\pgfline{\pgfxy(33.0,31.0)}{\pgfxy(33.0,10.0)}
\pgfline{\pgfxy(33.0,45.0)}{\pgfxy(33.0,66.0)}
\pgfsetlinewidth{0.33pt}
\pgfcircle[fill]{\pgfxy(33,10)}{1.4pt}\pgfcircle[fill]{\pgfxy(33,66)}{1.4pt}\pgfsetlinewidth{0.7pt}
\pgfline{\pgfxy(117.0,17.0)}{\pgfxy(124.0,17.0)}
\pgfline{\pgfxy(124.0,17.0)}{\pgfxy(124.0,31.0)}
\pgfline{\pgfxy(124.0,31.0)}{\pgfxy(131.0,31.0)}
\pgfline{\pgfxy(117.0,59.0)}{\pgfxy(124.0,59.0)}
\pgfline{\pgfxy(124.0,59.0)}{\pgfxy(124.0,45.0)}
\pgfline{\pgfxy(124.0,45.0)}{\pgfxy(131.0,45.0)}
\pgfline{\pgfxy(166.0,38.0)}{\pgfxy(187.0,38.0)}
\pgfline{\pgfxy(117.0,101.0)}{\pgfxy(187.0,101.0)}
\begin{pgfmagnify}{1}{-1}
\pgfputat{\pgfxy(5,-6)}{\pgfbox[left,top]{A}}
\end{pgfmagnify}
\begin{pgfmagnify}{1}{-1}
\pgfputat{\pgfxy(5,-62)}{\pgfbox[left,top]{B}}
\end{pgfmagnify}
\begin{pgfmagnify}{1}{-1}
\pgfputat{\pgfxy(190,-34)}{\pgfbox[left,top]{Sum}}
\end{pgfmagnify}
\begin{pgfmagnify}{1}{-1}
\pgfputat{\pgfxy(190,-97)}{\pgfbox[left,top]{Carry}}
\end{pgfmagnify}
\pgfline{\pgfxy(68.0,38.0)}{\pgfxy(68.0,101.0)}
\pgfsetlinewidth{0.33pt}
\pgfcircle[fill]{\pgfxy(68,38)}{1.4pt}\end{pgfmagnify}
\end{pgfpicture}}}
	\caption{Half Adder ND2 implementation}
	\label{fig:half_adder}
\end{figure}

\begin{figure}[H]
	\centering
	\textsf{{\fontsize{9pt}{4.0pt}
			\begin{pgfpicture}{0cm}{0cm}{383pt}{141pt}
% Created by FidoCadJ ver. 0.24.6, export filter by Davide Bucci
\pgfsetxvec{\pgfpoint{1pt}{0pt}}
\pgfsetyvec{\pgfpoint{0pt}{1pt}}
\pgfsetroundjoin 
\pgfsetroundcap
\pgftranslateto{\pgfxy(0,141)}
\begin{pgfmagnify}{1}{-1}
% Layer color definitions
\definecolor{layer0}{rgb}{0.0,0.0,0.0}
\definecolor{layer1}{rgb}{0.0,0.0,0.5}
\definecolor{layer2}{rgb}{1.0,0.0,0.0}
\definecolor{layer3}{rgb}{0.0,0.5,0.5}
\definecolor{layer4}{rgb}{1.0,0.78,0.0}
\definecolor{layer5}{rgb}{0.5,1.0,0.0}
\definecolor{layer6}{rgb}{0.0,1.0,1.0}
\definecolor{layer7}{rgb}{0.0,0.5,0.0}
\definecolor{layer8}{rgb}{0.6,0.8,0.2}
\definecolor{layer9}{rgb}{1.0,0.08,0.58}
\definecolor{layer10}{rgb}{0.71,0.61,0.05}
\definecolor{layer11}{rgb}{0.0,0.5,1.0}
\definecolor{layer12}{rgb}{0.88,0.88,0.88}
\definecolor{layer13}{rgb}{0.64,0.64,0.64}
\definecolor{layer14}{rgb}{0.37,0.37,0.37}
\definecolor{layer15}{rgb}{0.0,0.0,0.0}
% End of color definitions
\color{layer0}
\pgfsetlinewidth{0.7pt}
\pgfline{\pgfxy(40.0,31.0)}{\pgfxy(47.0,31.0)}
\pgfline{\pgfxy(47.0,28.0)}{\pgfxy(47.0,48.0)}
\pgfline{\pgfxy(40.0,45.0)}{\pgfxy(47.0,45.0)}
\pgfline{\pgfxy(47.0,48.0)}{\pgfxy(58.0,48.0)}
\pgfline{\pgfxy(47.0,28.0)}{\pgfxy(58.0,28.0)}
\pgfmoveto{\pgfxy(58,28)} 
\pgfcurveto{\pgfxy(71,28)}{\pgfxy(71,48)}{\pgfxy(58,48)}
\pgfstroke
\pgfellipse[stroke]{\pgfxy(69.5,38.0)}{\pgfxy(1.5,0)}{\pgfxy(0,1.0)}
\pgfline{\pgfxy(71.0,38.0)}{\pgfxy(75.0,38.0)}
\pgfline{\pgfxy(89.0,10.0)}{\pgfxy(96.0,10.0)}
\pgfline{\pgfxy(96.0,7.0)}{\pgfxy(96.0,27.0)}
\pgfline{\pgfxy(89.0,24.0)}{\pgfxy(96.0,24.0)}
\pgfline{\pgfxy(96.0,27.0)}{\pgfxy(107.0,27.0)}
\pgfline{\pgfxy(96.0,7.0)}{\pgfxy(107.0,7.0)}
\pgfmoveto{\pgfxy(107,7)} 
\pgfcurveto{\pgfxy(120,7)}{\pgfxy(120,27)}{\pgfxy(107,27)}
\pgfstroke
\pgfellipse[stroke]{\pgfxy(118.5,17.0)}{\pgfxy(1.5,0)}{\pgfxy(0,1.0)}
\pgfline{\pgfxy(120.0,17.0)}{\pgfxy(124.0,17.0)}
\pgfline{\pgfxy(89.0,52.0)}{\pgfxy(96.0,52.0)}
\pgfline{\pgfxy(96.0,49.0)}{\pgfxy(96.0,69.0)}
\pgfline{\pgfxy(89.0,66.0)}{\pgfxy(96.0,66.0)}
\pgfline{\pgfxy(96.0,69.0)}{\pgfxy(107.0,69.0)}
\pgfline{\pgfxy(96.0,49.0)}{\pgfxy(107.0,49.0)}
\pgfmoveto{\pgfxy(107,49)} 
\pgfcurveto{\pgfxy(120,49)}{\pgfxy(120,69)}{\pgfxy(107,69)}
\pgfstroke
\pgfellipse[stroke]{\pgfxy(118.5,59.0)}{\pgfxy(1.5,0)}{\pgfxy(0,1.0)}
\pgfline{\pgfxy(120.0,59.0)}{\pgfxy(124.0,59.0)}
\pgfline{\pgfxy(138.0,31.0)}{\pgfxy(145.0,31.0)}
\pgfline{\pgfxy(145.0,28.0)}{\pgfxy(145.0,48.0)}
\pgfline{\pgfxy(138.0,45.0)}{\pgfxy(145.0,45.0)}
\pgfline{\pgfxy(145.0,48.0)}{\pgfxy(156.0,48.0)}
\pgfline{\pgfxy(145.0,28.0)}{\pgfxy(156.0,28.0)}
\pgfmoveto{\pgfxy(156,28)} 
\pgfcurveto{\pgfxy(169,28)}{\pgfxy(169,48)}{\pgfxy(156,48)}
\pgfstroke
\pgfellipse[stroke]{\pgfxy(167.5,38.0)}{\pgfxy(1.5,0)}{\pgfxy(0,1.0)}
\pgfline{\pgfxy(169.0,38.0)}{\pgfxy(173.0,38.0)}
\pgfline{\pgfxy(75.0,38.0)}{\pgfxy(82.0,38.0)}
\pgfline{\pgfxy(82.0,38.0)}{\pgfxy(82.0,24.0)}
\pgfline{\pgfxy(82.0,24.0)}{\pgfxy(89.0,24.0)}
\pgfline{\pgfxy(82.0,38.0)}{\pgfxy(82.0,52.0)}
\pgfline{\pgfxy(82.0,52.0)}{\pgfxy(89.0,52.0)}
\pgfsetlinewidth{0.33pt}
\pgfcircle[fill]{\pgfxy(82,38)}{1.4pt}\pgfsetlinewidth{0.7pt}
\pgfline{\pgfxy(19.0,10.0)}{\pgfxy(89.0,10.0)}
\pgfline{\pgfxy(19.0,66.0)}{\pgfxy(89.0,66.0)}
\pgfline{\pgfxy(40.0,31.0)}{\pgfxy(40.0,10.0)}
\pgfline{\pgfxy(40.0,45.0)}{\pgfxy(40.0,66.0)}
\pgfsetlinewidth{0.33pt}
\pgfcircle[fill]{\pgfxy(40,10)}{1.4pt}\pgfcircle[fill]{\pgfxy(40,66)}{1.4pt}\pgfsetlinewidth{0.7pt}
\pgfline{\pgfxy(124.0,17.0)}{\pgfxy(131.0,17.0)}
\pgfline{\pgfxy(131.0,17.0)}{\pgfxy(131.0,31.0)}
\pgfline{\pgfxy(131.0,31.0)}{\pgfxy(138.0,31.0)}
\pgfline{\pgfxy(124.0,59.0)}{\pgfxy(131.0,59.0)}
\pgfline{\pgfxy(131.0,59.0)}{\pgfxy(131.0,45.0)}
\pgfline{\pgfxy(131.0,45.0)}{\pgfxy(138.0,45.0)}
\pgfline{\pgfxy(173.0,38.0)}{\pgfxy(250.0,38.0)}
\begin{pgfmagnify}{1}{-1}
\pgfputat{\pgfxy(12,-6)}{\pgfbox[left,top]{A}}
\end{pgfmagnify}
\begin{pgfmagnify}{1}{-1}
\pgfputat{\pgfxy(12,-62)}{\pgfbox[left,top]{B}}
\end{pgfmagnify}
\pgfline{\pgfxy(250.0,38.0)}{\pgfxy(257.0,38.0)}
\pgfline{\pgfxy(257.0,35.0)}{\pgfxy(257.0,55.0)}
\pgfline{\pgfxy(250.0,52.0)}{\pgfxy(257.0,52.0)}
\pgfline{\pgfxy(257.0,55.0)}{\pgfxy(268.0,55.0)}
\pgfline{\pgfxy(257.0,35.0)}{\pgfxy(268.0,35.0)}
\pgfmoveto{\pgfxy(268,35)} 
\pgfcurveto{\pgfxy(281,35)}{\pgfxy(281,55)}{\pgfxy(268,55)}
\pgfstroke
\pgfellipse[stroke]{\pgfxy(279.5,45.0)}{\pgfxy(1.5,0)}{\pgfxy(0,1.0)}
\pgfline{\pgfxy(281.0,45.0)}{\pgfxy(285.0,45.0)}
\pgfline{\pgfxy(201.0,59.0)}{\pgfxy(208.0,59.0)}
\pgfline{\pgfxy(208.0,56.0)}{\pgfxy(208.0,76.0)}
\pgfline{\pgfxy(201.0,73.0)}{\pgfxy(208.0,73.0)}
\pgfline{\pgfxy(208.0,76.0)}{\pgfxy(219.0,76.0)}
\pgfline{\pgfxy(208.0,56.0)}{\pgfxy(219.0,56.0)}
\pgfmoveto{\pgfxy(219,56)} 
\pgfcurveto{\pgfxy(232,56)}{\pgfxy(232,76)}{\pgfxy(219,76)}
\pgfstroke
\pgfellipse[stroke]{\pgfxy(230.5,66.0)}{\pgfxy(1.5,0)}{\pgfxy(0,1.0)}
\pgfline{\pgfxy(232.0,66.0)}{\pgfxy(236.0,66.0)}
\pgfline{\pgfxy(250.0,80.0)}{\pgfxy(257.0,80.0)}
\pgfline{\pgfxy(257.0,77.0)}{\pgfxy(257.0,97.0)}
\pgfline{\pgfxy(250.0,94.0)}{\pgfxy(257.0,94.0)}
\pgfline{\pgfxy(257.0,97.0)}{\pgfxy(268.0,97.0)}
\pgfline{\pgfxy(257.0,77.0)}{\pgfxy(268.0,77.0)}
\pgfmoveto{\pgfxy(268,77)} 
\pgfcurveto{\pgfxy(281,77)}{\pgfxy(281,97)}{\pgfxy(268,97)}
\pgfstroke
\pgfellipse[stroke]{\pgfxy(279.5,87.0)}{\pgfxy(1.5,0)}{\pgfxy(0,1.0)}
\pgfline{\pgfxy(281.0,87.0)}{\pgfxy(285.0,87.0)}
\pgfline{\pgfxy(299.0,59.0)}{\pgfxy(306.0,59.0)}
\pgfline{\pgfxy(306.0,56.0)}{\pgfxy(306.0,76.0)}
\pgfline{\pgfxy(299.0,73.0)}{\pgfxy(306.0,73.0)}
\pgfline{\pgfxy(306.0,76.0)}{\pgfxy(317.0,76.0)}
\pgfline{\pgfxy(306.0,56.0)}{\pgfxy(317.0,56.0)}
\pgfmoveto{\pgfxy(317,56)} 
\pgfcurveto{\pgfxy(330,56)}{\pgfxy(330,76)}{\pgfxy(317,76)}
\pgfstroke
\pgfellipse[stroke]{\pgfxy(328.5,66.0)}{\pgfxy(1.5,0)}{\pgfxy(0,1.0)}
\pgfline{\pgfxy(330.0,66.0)}{\pgfxy(334.0,66.0)}
\pgfline{\pgfxy(236.0,66.0)}{\pgfxy(243.0,66.0)}
\pgfline{\pgfxy(243.0,66.0)}{\pgfxy(243.0,52.0)}
\pgfline{\pgfxy(243.0,52.0)}{\pgfxy(250.0,52.0)}
\pgfline{\pgfxy(243.0,66.0)}{\pgfxy(243.0,80.0)}
\pgfline{\pgfxy(243.0,80.0)}{\pgfxy(250.0,80.0)}
\pgfsetlinewidth{0.33pt}
\pgfcircle[fill]{\pgfxy(243,66)}{1.4pt}\pgfsetlinewidth{0.7pt}
\pgfline{\pgfxy(285.0,45.0)}{\pgfxy(292.0,45.0)}
\pgfline{\pgfxy(292.0,45.0)}{\pgfxy(292.0,59.0)}
\pgfline{\pgfxy(292.0,59.0)}{\pgfxy(299.0,59.0)}
\pgfline{\pgfxy(285.0,87.0)}{\pgfxy(292.0,87.0)}
\pgfline{\pgfxy(292.0,87.0)}{\pgfxy(292.0,73.0)}
\pgfline{\pgfxy(292.0,73.0)}{\pgfxy(299.0,73.0)}
\pgfline{\pgfxy(201.0,59.0)}{\pgfxy(194.0,59.0)}
\pgfline{\pgfxy(194.0,59.0)}{\pgfxy(194.0,38.0)}
\pgfline{\pgfxy(250.0,94.0)}{\pgfxy(19.0,94.0)}
\pgfline{\pgfxy(201.0,73.0)}{\pgfxy(194.0,73.0)}
\pgfline{\pgfxy(194.0,73.0)}{\pgfxy(194.0,94.0)}
\pgfsetlinewidth{0.33pt}
\pgfcircle[fill]{\pgfxy(194,38)}{1.4pt}\pgfcircle[fill]{\pgfxy(194,94)}{1.4pt}\pgfsetlinewidth{0.7pt}
\pgfline{\pgfxy(250.0,122.0)}{\pgfxy(257.0,122.0)}
\pgfline{\pgfxy(257.0,119.0)}{\pgfxy(257.0,139.0)}
\pgfline{\pgfxy(250.0,136.0)}{\pgfxy(257.0,136.0)}
\pgfline{\pgfxy(257.0,139.0)}{\pgfxy(268.0,139.0)}
\pgfline{\pgfxy(257.0,119.0)}{\pgfxy(268.0,119.0)}
\pgfmoveto{\pgfxy(268,119)} 
\pgfcurveto{\pgfxy(281,119)}{\pgfxy(281,139)}{\pgfxy(268,139)}
\pgfstroke
\pgfellipse[stroke]{\pgfxy(279.5,129.0)}{\pgfxy(1.5,0)}{\pgfxy(0,1.0)}
\pgfline{\pgfxy(281.0,129.0)}{\pgfxy(285.0,129.0)}
\pgfline{\pgfxy(236.0,66.0)}{\pgfxy(236.0,122.0)}
\pgfline{\pgfxy(236.0,122.0)}{\pgfxy(250.0,122.0)}
\pgfline{\pgfxy(75.0,38.0)}{\pgfxy(75.0,136.0)}
\pgfline{\pgfxy(75.0,136.0)}{\pgfxy(250.0,136.0)}
\pgfline{\pgfxy(334.0,66.0)}{\pgfxy(355.0,66.0)}
\pgfline{\pgfxy(285.0,129.0)}{\pgfxy(355.0,129.0)}
\begin{pgfmagnify}{1}{-1}
\pgfputat{\pgfxy(356,-62)}{\pgfbox[left,top]{Sum}}
\end{pgfmagnify}
\begin{pgfmagnify}{1}{-1}
\pgfputat{\pgfxy(356,-125)}{\pgfbox[left,top]{Carry}}
\end{pgfmagnify}
\begin{pgfmagnify}{1}{-1}
\pgfputat{\pgfxy(5,-90)}{\pgfbox[left,top]{Cin}}
\end{pgfmagnify}
\pgfsetlinewidth{0.33pt}
\pgfcircle[fill]{\pgfxy(75,38)}{1.4pt}\pgfcircle[fill]{\pgfxy(236,66)}{1.4pt}\end{pgfmagnify}
\end{pgfpicture}}}
	\caption{Full Adder ND2 implementation}
	\label{fig:full_adder}
\end{figure}

\begin{figure}[H]
	\centering
	\textsf{{\fontsize{9pt}{5.5pt}
			\begin{pgfpicture}{0cm}{0cm}{301pt}{81pt}
% Created by FidoCadJ ver. 0.24.7, export filter by Davide Bucci
\pgfsetxvec{\pgfpoint{1pt}{0pt}}
\pgfsetyvec{\pgfpoint{0pt}{1pt}}
\pgfsetroundjoin 
\pgfsetroundcap
\pgftranslateto{\pgfxy(0,81)}
\begin{pgfmagnify}{1}{-1}
% Layer color definitions
\definecolor{layer0}{rgb}{0.0,0.0,0.0}
\definecolor{layer1}{rgb}{0.0,0.0,0.5}
\definecolor{layer2}{rgb}{1.0,0.0,0.0}
\definecolor{layer3}{rgb}{0.0,0.5,0.5}
\definecolor{layer4}{rgb}{1.0,0.78,0.0}
\definecolor{layer5}{rgb}{0.5,1.0,0.0}
\definecolor{layer6}{rgb}{0.0,1.0,1.0}
\definecolor{layer7}{rgb}{0.0,0.5,0.0}
\definecolor{layer8}{rgb}{0.6,0.8,0.2}
\definecolor{layer9}{rgb}{1.0,0.08,0.58}
\definecolor{layer10}{rgb}{0.71,0.61,0.05}
\definecolor{layer11}{rgb}{0.0,0.5,1.0}
\definecolor{layer12}{rgb}{0.88,0.88,0.88}
\definecolor{layer13}{rgb}{0.64,0.64,0.64}
\definecolor{layer14}{rgb}{0.37,0.37,0.37}
\definecolor{layer15}{rgb}{0.0,0.0,0.0}
% End of color definitions
\color{layer0}
\pgfsetlinewidth{0.5pt}
\pgfline{\pgfxy(118.0,13.0)}{\pgfxy(123.0,13.0)}
\pgfline{\pgfxy(123.0,11.0)}{\pgfxy(123.0,25.0)}
\pgfline{\pgfxy(118.0,23.0)}{\pgfxy(123.0,23.0)}
\pgfline{\pgfxy(123.0,25.0)}{\pgfxy(131.0,25.0)}
\pgfline{\pgfxy(123.0,11.0)}{\pgfxy(131.0,11.0)}
\pgfmoveto{\pgfxy(131,11)} 
\pgfcurveto{\pgfxy(140,11)}{\pgfxy(140,25)}{\pgfxy(131,25)}
\pgfstroke
\pgfellipse[stroke]{\pgfxy(139.0,18.0)}{\pgfxy(1.0,0)}{\pgfxy(0,1.0)}
\pgfline{\pgfxy(140.0,18.0)}{\pgfxy(143.0,18.0)}
\pgfline{\pgfxy(118.0,43.0)}{\pgfxy(123.0,43.0)}
\pgfline{\pgfxy(123.0,41.0)}{\pgfxy(123.0,55.0)}
\pgfline{\pgfxy(118.0,53.0)}{\pgfxy(123.0,53.0)}
\pgfline{\pgfxy(123.0,55.0)}{\pgfxy(131.0,55.0)}
\pgfline{\pgfxy(123.0,41.0)}{\pgfxy(131.0,41.0)}
\pgfmoveto{\pgfxy(131,41)} 
\pgfcurveto{\pgfxy(140,41)}{\pgfxy(140,55)}{\pgfxy(131,55)}
\pgfstroke
\pgfellipse[stroke]{\pgfxy(139.0,48.0)}{\pgfxy(1.0,0)}{\pgfxy(0,1.0)}
\pgfline{\pgfxy(140.0,48.0)}{\pgfxy(143.0,48.0)}
\pgfline{\pgfxy(143.0,18.0)}{\pgfxy(148.0,18.0)}
\pgfline{\pgfxy(143.0,48.0)}{\pgfxy(148.0,48.0)}
\pgfline{\pgfxy(148.0,48.0)}{\pgfxy(148.0,43.0)}
\pgfline{\pgfxy(148.0,43.0)}{\pgfxy(113.0,28.0)}
\pgfline{\pgfxy(113.0,28.0)}{\pgfxy(113.0,23.0)}
\pgfline{\pgfxy(113.0,23.0)}{\pgfxy(118.0,23.0)}
\pgfline{\pgfxy(148.0,18.0)}{\pgfxy(148.0,23.0)}
\pgfline{\pgfxy(148.0,23.0)}{\pgfxy(113.0,38.0)}
\pgfline{\pgfxy(113.0,38.0)}{\pgfxy(113.0,43.0)}
\pgfline{\pgfxy(113.0,43.0)}{\pgfxy(118.0,43.0)}
\pgfsetlinewidth{0.33pt}
\pgfcircle[fill]{\pgfxy(178,33)}{1.0pt}\pgfsetlinewidth{0.5pt}
\pgfline{\pgfxy(183.0,8.0)}{\pgfxy(188.0,8.0)}
\pgfline{\pgfxy(188.0,6.0)}{\pgfxy(188.0,20.0)}
\pgfline{\pgfxy(183.0,18.0)}{\pgfxy(188.0,18.0)}
\pgfline{\pgfxy(188.0,20.0)}{\pgfxy(196.0,20.0)}
\pgfline{\pgfxy(188.0,6.0)}{\pgfxy(196.0,6.0)}
\pgfmoveto{\pgfxy(196,6)} 
\pgfcurveto{\pgfxy(205,6)}{\pgfxy(205,20)}{\pgfxy(196,20)}
\pgfstroke
\pgfellipse[stroke]{\pgfxy(204.0,13.0)}{\pgfxy(1.0,0)}{\pgfxy(0,1.0)}
\pgfline{\pgfxy(205.0,13.0)}{\pgfxy(208.0,13.0)}
\pgfline{\pgfxy(178.0,18.0)}{\pgfxy(183.0,18.0)}
\pgfline{\pgfxy(178.0,33.0)}{\pgfxy(178.0,18.0)}
\pgfline{\pgfxy(183.0,48.0)}{\pgfxy(188.0,48.0)}
\pgfline{\pgfxy(188.0,46.0)}{\pgfxy(188.0,60.0)}
\pgfline{\pgfxy(183.0,58.0)}{\pgfxy(188.0,58.0)}
\pgfline{\pgfxy(188.0,60.0)}{\pgfxy(196.0,60.0)}
\pgfline{\pgfxy(188.0,46.0)}{\pgfxy(196.0,46.0)}
\pgfmoveto{\pgfxy(196,46)} 
\pgfcurveto{\pgfxy(205,46)}{\pgfxy(205,60)}{\pgfxy(196,60)}
\pgfstroke
\pgfellipse[stroke]{\pgfxy(204.0,53.0)}{\pgfxy(1.0,0)}{\pgfxy(0,1.0)}
\pgfline{\pgfxy(205.0,53.0)}{\pgfxy(208.0,53.0)}
\pgfline{\pgfxy(178.0,48.0)}{\pgfxy(183.0,48.0)}
\pgfline{\pgfxy(178.0,33.0)}{\pgfxy(178.0,48.0)}
\pgfline{\pgfxy(148.0,18.0)}{\pgfxy(148.0,8.0)}
\pgfline{\pgfxy(148.0,8.0)}{\pgfxy(183.0,8.0)}
\pgfline{\pgfxy(148.0,48.0)}{\pgfxy(148.0,58.0)}
\pgfline{\pgfxy(148.0,58.0)}{\pgfxy(183.0,58.0)}
\pgfline{\pgfxy(178.0,33.0)}{\pgfxy(163.0,33.0)}
\pgfsetlinewidth{0.33pt}
\pgfcircle[fill]{\pgfxy(148,48)}{1.0pt}\pgfcircle[fill]{\pgfxy(148,18)}{1.0pt}\pgfsetlinewidth{0.5pt}
\pgfline{\pgfxy(228.0,13.0)}{\pgfxy(233.0,13.0)}
\pgfline{\pgfxy(233.0,11.0)}{\pgfxy(233.0,25.0)}
\pgfline{\pgfxy(228.0,23.0)}{\pgfxy(233.0,23.0)}
\pgfline{\pgfxy(233.0,25.0)}{\pgfxy(241.0,25.0)}
\pgfline{\pgfxy(233.0,11.0)}{\pgfxy(241.0,11.0)}
\pgfmoveto{\pgfxy(241,11)} 
\pgfcurveto{\pgfxy(250,11)}{\pgfxy(250,25)}{\pgfxy(241,25)}
\pgfstroke
\pgfellipse[stroke]{\pgfxy(249.0,18.0)}{\pgfxy(1.0,0)}{\pgfxy(0,1.0)}
\pgfline{\pgfxy(250.0,18.0)}{\pgfxy(253.0,18.0)}
\pgfline{\pgfxy(228.0,43.0)}{\pgfxy(233.0,43.0)}
\pgfline{\pgfxy(233.0,41.0)}{\pgfxy(233.0,55.0)}
\pgfline{\pgfxy(228.0,53.0)}{\pgfxy(233.0,53.0)}
\pgfline{\pgfxy(233.0,55.0)}{\pgfxy(241.0,55.0)}
\pgfline{\pgfxy(233.0,41.0)}{\pgfxy(241.0,41.0)}
\pgfmoveto{\pgfxy(241,41)} 
\pgfcurveto{\pgfxy(250,41)}{\pgfxy(250,55)}{\pgfxy(241,55)}
\pgfstroke
\pgfellipse[stroke]{\pgfxy(249.0,48.0)}{\pgfxy(1.0,0)}{\pgfxy(0,1.0)}
\pgfline{\pgfxy(250.0,48.0)}{\pgfxy(253.0,48.0)}
\pgfline{\pgfxy(253.0,18.0)}{\pgfxy(258.0,18.0)}
\pgfline{\pgfxy(253.0,48.0)}{\pgfxy(258.0,48.0)}
\pgfline{\pgfxy(258.0,48.0)}{\pgfxy(258.0,43.0)}
\pgfline{\pgfxy(258.0,43.0)}{\pgfxy(223.0,28.0)}
\pgfline{\pgfxy(223.0,28.0)}{\pgfxy(223.0,23.0)}
\pgfline{\pgfxy(223.0,23.0)}{\pgfxy(228.0,23.0)}
\pgfline{\pgfxy(258.0,18.0)}{\pgfxy(258.0,23.0)}
\pgfline{\pgfxy(258.0,23.0)}{\pgfxy(223.0,38.0)}
\pgfline{\pgfxy(223.0,38.0)}{\pgfxy(223.0,43.0)}
\pgfline{\pgfxy(223.0,43.0)}{\pgfxy(228.0,43.0)}
\pgfline{\pgfxy(258.0,18.0)}{\pgfxy(273.0,18.0)}
\pgfsetlinewidth{0.33pt}
\pgfcircle[fill]{\pgfxy(258,18)}{1.0pt}\pgfcircle[fill]{\pgfxy(258,48)}{1.0pt}\pgfsetlinewidth{0.5pt}
\pgfline{\pgfxy(208.0,13.0)}{\pgfxy(228.0,13.0)}
\pgfline{\pgfxy(208.0,53.0)}{\pgfxy(228.0,53.0)}
\pgfsetlinewidth{0.33pt}
\pgfcircle[fill]{\pgfxy(73,33)}{1.0pt}\pgfsetlinewidth{0.5pt}
\pgfline{\pgfxy(78.0,8.0)}{\pgfxy(83.0,8.0)}
\pgfline{\pgfxy(83.0,6.0)}{\pgfxy(83.0,20.0)}
\pgfline{\pgfxy(78.0,18.0)}{\pgfxy(83.0,18.0)}
\pgfline{\pgfxy(83.0,20.0)}{\pgfxy(91.0,20.0)}
\pgfline{\pgfxy(83.0,6.0)}{\pgfxy(91.0,6.0)}
\pgfmoveto{\pgfxy(91,6)} 
\pgfcurveto{\pgfxy(100,6)}{\pgfxy(100,20)}{\pgfxy(91,20)}
\pgfstroke
\pgfellipse[stroke]{\pgfxy(99.0,13.0)}{\pgfxy(1.0,0)}{\pgfxy(0,1.0)}
\pgfline{\pgfxy(100.0,13.0)}{\pgfxy(103.0,13.0)}
\pgfline{\pgfxy(73.0,18.0)}{\pgfxy(78.0,18.0)}
\pgfline{\pgfxy(73.0,33.0)}{\pgfxy(73.0,18.0)}
\pgfline{\pgfxy(78.0,48.0)}{\pgfxy(83.0,48.0)}
\pgfline{\pgfxy(83.0,46.0)}{\pgfxy(83.0,60.0)}
\pgfline{\pgfxy(78.0,58.0)}{\pgfxy(83.0,58.0)}
\pgfline{\pgfxy(83.0,60.0)}{\pgfxy(91.0,60.0)}
\pgfline{\pgfxy(83.0,46.0)}{\pgfxy(91.0,46.0)}
\pgfmoveto{\pgfxy(91,46)} 
\pgfcurveto{\pgfxy(100,46)}{\pgfxy(100,60)}{\pgfxy(91,60)}
\pgfstroke
\pgfellipse[stroke]{\pgfxy(99.0,53.0)}{\pgfxy(1.0,0)}{\pgfxy(0,1.0)}
\pgfline{\pgfxy(100.0,53.0)}{\pgfxy(103.0,53.0)}
\pgfline{\pgfxy(73.0,48.0)}{\pgfxy(78.0,48.0)}
\pgfline{\pgfxy(73.0,33.0)}{\pgfxy(73.0,48.0)}
\pgfline{\pgfxy(61.0,33.0)}{\pgfxy(63.0,33.0)}
\pgfellipse[stroke]{\pgfxy(60.0,33.0)}{\pgfxy(1.0,0)}{\pgfxy(0,1.0)}
\pgfmoveto{\pgfxy(49.0,28.0)}
\pgflineto{\pgfxy(49.0,38.0)}
\pgflineto{\pgfxy(59.0,33.0)}
\pgfclosepath 
\pgfqstroke 
\pgfline{\pgfxy(43.0,33.0)}{\pgfxy(49.0,33.0)}
\pgfline{\pgfxy(73.0,33.0)}{\pgfxy(63.0,33.0)}
\pgfsetlinewidth{0.33pt}
\pgfcircle[fill]{\pgfxy(38,33)}{1.0pt}\pgfsetlinewidth{0.5pt}
\pgfline{\pgfxy(103.0,13.0)}{\pgfxy(118.0,13.0)}
\pgfline{\pgfxy(98.0,53.0)}{\pgfxy(118.0,53.0)}
\pgfline{\pgfxy(38.0,78.0)}{\pgfxy(163.0,78.0)}
\pgfline{\pgfxy(163.0,33.0)}{\pgfxy(163.0,78.0)}
\pgfline{\pgfxy(38.0,33.0)}{\pgfxy(38.0,78.0)}
\pgfline{\pgfxy(43.0,8.0)}{\pgfxy(43.0,58.0)}
\pgfsetlinewidth{0.33pt}
\pgfcircle[fill]{\pgfxy(43,8)}{1.0pt}\pgfsetlinewidth{0.5pt}
\pgfline{\pgfxy(76.0,58.0)}{\pgfxy(78.0,58.0)}
\pgfellipse[stroke]{\pgfxy(75.0,58.0)}{\pgfxy(1.0,0)}{\pgfxy(0,1.0)}
\pgfmoveto{\pgfxy(64.0,53.0)}
\pgflineto{\pgfxy(64.0,63.0)}
\pgflineto{\pgfxy(74.0,58.0)}
\pgfclosepath 
\pgfqstroke 
\pgfline{\pgfxy(58.0,58.0)}{\pgfxy(64.0,58.0)}
\pgfline{\pgfxy(43.0,58.0)}{\pgfxy(58.0,58.0)}
\begin{pgfmagnify}{1}{-1}
\pgfputat{\pgfxy(278,-13)}{\pgfbox[left,top]{Q}}
\end{pgfmagnify}
\begin{pgfmagnify}{1}{-1}
\pgfputat{\pgfxy(278,-43)}{\pgfbox[left,top]{not(Q)}}
\end{pgfmagnify}
\pgfline{\pgfxy(258.0,48.0)}{\pgfxy(273.0,48.0)}
\begin{pgfmagnify}{1}{-1}
\pgfputat{\pgfxy(3,-28)}{\pgfbox[left,top]{Clock}}
\end{pgfmagnify}
\begin{pgfmagnify}{1}{-1}
\pgfputat{\pgfxy(13,-3)}{\pgfbox[left,top]{D}}
\end{pgfmagnify}
\pgfline{\pgfxy(23.0,8.0)}{\pgfxy(78.0,8.0)}
\pgfline{\pgfxy(43.0,33.0)}{\pgfxy(23.0,33.0)}
\end{pgfmagnify}
\end{pgfpicture}}}
	\caption{D flip flop implementation}
	\label{fig:f_f}
\end{figure}
The code requires as inputs the ND2 area and the INV one: the total area has been calculated as follows:

\begin{equation}
area_{FA}=9\cdot area_{ND2}
\end{equation}
\begin{equation}
area_{HA}=5\cdot area_{ND2}
\end{equation}
\begin{equation}
area_{FF}=8\cdot area_{ND2}+2\cdot area_{INV}
\end{equation}
\begin{equation}
area_{total}=N_{FA}\cdot area_{FA}+N_{HA}\cdot area_{HA}+N_{FF}\cdot area_{FF}
\end{equation}

$N_{FA}$, $N_{HA}$ and $N_{FF}$ are respectively referred to the number of full adders, half adders and flip flops instantiated in the Wallace structure.