\subsection{Timing analysis}
In order to estimate the maximum operating frequency the critical path has been calculated. In a pipelined version of the Wallace Tree, the critical path to be considered is the longest combinatorial path between 2 layers of flip flops, in particular it is represented by the path which goes through a certain number of full adders (that depends on the number of levels of the tree and the number of pipeline stages inserted).

Since the system is timed through flip flops, the Clock-to-Output delay and set-up time have been also considered. In particular the minimum allowed clock period (in the fully pipelined case) will therefore be:

\begin{equation}
t_{min}=t_{ClkToOut}+delay_{FA}+t_{setup}
\end{equation}

Otherwise, the number of FA between two layers of Flip-Flops has to be considered. The full adder delay is given by the following expression:
\begin{equation}
delay_{FA}=6\cdot\tau_{ND2}
\end{equation}
$t_{ClkToOut}$ and $t_{setup}$ are passed as inputs to the function implemented for the estimations.
The delay of a single ND2 gate, $\tau_{ND2}$, is calculated by considering the model of delay in the Roadmap, that is the following:
\begin{equation}
\tau_{ND2}=\frac{C_{ND2}\cdot V_{DD}}{I_{ND2}}
\end{equation}
In the Matlab function that has been implemented $\tau_{ND2}$ is passed as an input as well.

Finally the Wallace Tree delay has been obtained as:
\begin{equation}
delay_{total}=N_{pipestage}\cdot t_{min}
\end{equation}

Once obtained the critical path, it is possible to evaluate the maximum operating frequency, $f_{max}$, calculated as the reverse of $t_{min}$. 

In the case of a non-pipelined implemetation the delay is given by $N$ times the FA delay, where $N$ is the number of levels of the tree.